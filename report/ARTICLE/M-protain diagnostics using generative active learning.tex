\documentclass[letterpaper]{article}
\usepackage{aaai20}
\usepackage{times}
\usepackage{helvet}
\usepackage{courier}
\usepackage[hyphens]{url}
\usepackage{graphicx}
\urlstyle{rm}
\def\UrlFont{\rm}
\usepackage{graphicx}
\frenchspacing
\setlength{\pdfpagewidth}{8.5in}
\setlength{\pdfpageheight}{11in}
% Add additional packages here, but check
% the list of disallowed packages
% (including, but not limited to
% authblk, caption, CJK, float, fullpage, geometry,
% hyperref, layout, nameref, natbib, savetrees,
% setspace, titlesec, tocbibind, ulem)
% and illegal commands provided in the
% common formatting errors document
% included in the  Author Kit before doing so.
%
% PDFINFO
% You are required to complete the following
% for pass-through to the PDF.
% No LaTeX commands of any kind may be
% entered. The parentheses and spaces
% are an integral part of the
% pdfinfo script and must not be removed.
%
\pdfinfo{
/Title (M-Protain Diagnostics Using Generative Active Learning)
/Author (Hanyu Li, Junsong Yuan)
/Keywords (Active Learning, Generative Adversarial Nets, M-Protain)
}
%
% Section Numbers
% Uncomment if you want to use section numbers
% and change the 0 to a 1 or 2
% \setcounter{secnumdepth}{0}
% Title and Author Information Must Immediately Follow
% the pdfinfo within the preamble
%


\title{M-Protain Diagnostics Using Generative Active Learning}


\author{Hanyu Li\\
Department of Physics\\
Tsinghua University\\
l-hy16@mails.tsinghua.edu.cn
\And
Junsong Yuan\\
Department of Computer Science and Engineering\\
University at Buffalo\\
jsyuan@buffalo.edu}


\begin{document}
\maketitle


\begin{abstract}
    With comparatively less labeled data and high labeling cost, most of the medical involved tasks can not be directly tackled by state of art machine learning approaches for their demand of large carefully labeled datasets.
\end{abstract}


\section{Introduction}
\par As deep models achieve astonishing results in almost every machine learning tasks, some unavoidable problems such as the need for large carefully labeled dataset has troubled researchers from the start. Part of the reason behind the tremendous success in deep learning is the availability of large-scale labeled data\cite{sun2017revisiting}. Although data labeling companies and platforms claim that they can provide inexpensive yet high quality data\cite{buhrmester2011amazon}, achieving such datasets can be extremely costly or even unrealistic in the scenarios where labeling requires high professionality. For instance, some medical image tasks can not be labeled by people without systematic training. However, the small number of these experts has determined that large-scale dataset is difficult to biuld. Plus, they are probably already preoccupied.

M-protain can be detected by immunofixation electrophoresis(IFE) images, and different categories of results may indicate MGUS, multiple myeloma, etc.

Active learning and GAN

Our paper try to 


\section{Related Work}
Multiple techniques were used to tackle with small datasets and partially labeled datasets including active learning\cite{settles2009active}, generative models\cite{goodfellow2014generative}\cite{kingma2013auto}, data augmentation\cite{tanner1987calculation}, and domain transfer\cite{pan2009survey} etc. asdlfdklflkdjflkadf;asdafsdjklfajfkjadf\cite{Zhu2017GenerativeAA}

\section{Preliminaries}
The problem is defined as below

\section{Proposed Method}
Our method make use of

\section{Experiments}
I did such experiments

\section{Discussion}
After I did these experiments

\section{Conclusion}
To sum up

% References and End of Paper
% These lines must be placed at the end of your paper
\bibliography{lib}
\bibliographystyle{aaai}
\end{document}